% This is part of the TFTB Reference Manual.
% Copyright (C) 1996 CNRS (France) and Rice University (US).
% See the file refguide.tex for copying conditions.



\markright{movsp2wv}
\hspace*{-1.6cm}{\Large \bf movsp2wv}

\vspace*{-.4cm}
\hspace*{-1.6cm}\rule[0in]{16.5cm}{.02cm}
\vspace*{.2cm}



{\bf \large \sf Purpose}\\
\hspace*{1.5cm}
\begin{minipage}[t]{13.5cm}
Movie illustrating the passage from the spectrogram to the WVD.
\end{minipage}
\vspace*{.5cm}


{\bf \large \sf Synopsis}\\
\hspace*{1.5cm}
\begin{minipage}[t]{13.5cm}
\begin{verbatim}
M = movsp2wv(N)
M = movsp2wv(N,Np)
\end{verbatim}
\end{minipage}
\vspace*{.5cm}


{\bf \large \sf Description}\\
\hspace*{1.5cm}
\begin{minipage}[t]{13.5cm}
        {\ty movsp2wv} generates the movie frames illustrating the passage
        from the spectrogram to the WVD using the smoothed pseudo-WVD with
        different smoothing gaussian windows. \\

\hspace*{-.5cm}\begin{tabular*}{14cm}{p{1.5cm} p{8.5cm} c}
Name & Description & Default value\\
\hline
        {\ty N}  & number of points of the analyzed signal\\
        {\ty Np} & number of snapshots & {\ty 8}\\
\hline  {\ty M} & matrix of movie frames\\
\hline
\end{tabular*}

\end{minipage}
\vspace*{1cm}


{\bf \large \sf Example}
\begin{verbatim}
         M=movsp2wv(128,15); 
         movie(M,10);
\end{verbatim}
\vspace*{.5cm}


{\bf \large \sf See Also}\\
\hspace*{1.5cm}
\begin{minipage}[t]{13.5cm}
\begin{verbatim}
movpwjph, movpwdph, movsc2wv, movcw4at, movwv2at.
\end{verbatim}
\end{minipage}
