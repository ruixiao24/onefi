% This is part of the TFTB Reference Manual.
% Copyright (C) 1996 CNRS (France) and Rice University (US).
% See the file refguide.tex for copying conditions.


\renewcommand{\footnoterule}{}
\markright{tfrbert}
\section*{\hspace*{-1.6cm} tfrbert}

\vspace*{-.4cm}
\hspace*{-1.6cm}\rule[0in]{16.5cm}{.02cm}
\vspace*{.2cm}

{\bf \large \sf Purpose}\\
\hspace*{1.5cm}
\begin{minipage}[t]{13.5cm}
Unitary Bertrand time-frequency distribution.
\end{minipage}
\vspace*{.5cm}

{\bf \large \sf Synopsis}\\
\hspace*{1.5cm}
\begin{minipage}[t]{13.5cm}
\begin{verbatim}
[tfr,t,f] = tfrbert(x)
[tfr,t,f] = tfrbert(x,t)
[tfr,t,f] = tfrbert(x,t,fmin,fmax)
[tfr,t,f] = tfrbert(x,t,fmin,fmax,N)
[tfr,t,f] = tfrbert(x,t,fmin,fmax,N,trace)
\end{verbatim}
\end{minipage}
\vspace*{.5cm}

{\bf \large \sf Description}\\
\hspace*{1.5cm}
\begin{minipage}[t]{13.5cm}
        {\ty tfrbert} generates the auto- or cross- unitary Bertrand
        distribution, defined as
\begin{eqnarray*}
B_x(t,\nu) =\nu \int_{-\infty}^{+\infty}
  \frac{u/2}{\sinh\left(\frac{u}{2}\right)}\ X\left(\frac{\nu\ u\
  e^{-u/2}}{2 \sinh\left(\frac{u}{2}\right)}\right)\ X^*\left(\frac{\nu\ u\
  e^{+u/2}}{2 \sinh\left(\frac{u}{2}\right)}\right)\ e^{-j2\pi\nu ut}\ du
\end{eqnarray*}
where $X(\nu)$ is the Fourier transform of $x(t)$.\\

\hspace*{-.5cm}\begin{tabular*}{14cm}{p{1.5cm} p{8.5cm} c}
Name & Description & Default value\\
\hline
        {\ty x} & signal (in time) to be analyzed. If {\ty x=[x1 x2]}, {\ty tfrbert} 
           computes the cross-unitary Bertrand distribution {\ty (Nx=length(x))}\\
        {\ty t} & time instant(s) on which the {\ty tfr} is evaluated & {\ty (1:Nx)}\\
        {\ty fmin, fmax} & respectively lower and upper frequency bounds of 
           the analyzed signal. These parameters fix the equivalent 
           frequency bandwidth (expressed in Hz). When unspecified, you
           have to enter them at the command line from the plot of the
           spectrum. {\ty fmin} and {\ty fmax} must be $>0$ and $\leq 0.5$\\
        {\ty N} & number of analyzed voices & auto\footnote{This value,
	determined from {\ty fmin} and {\ty fmax}, is the 
	next-power-of-two of the minimum value checking the non-overlapping
	condition in the fast Mellin transform.}\\
        {\ty trace} & if nonzero, the progression of the algorithm is shown
                                                & {\ty 0}\\

\hline\end{tabular*}\end{minipage} \newpage
\hspace*{1.5cm}\begin{minipage}[t]{13.5cm}
\hspace*{-.5cm}\begin{tabular*}{14cm}{p{1.5cm} p{8.5cm} c}
Name & Description & Default value\\\hline

     \hline {\ty tfr} & time-frequency matrix containing the coefficients of the
           distribution (x-coordinate corresponds to uniformly sampled 
           time, and y-coordinate corresponds to a geometrically sampled
           frequency). First row of {\ty tfr} corresponds to the lowest 
           frequency\\
        {\ty f} & vector of normalized frequencies (geometrically sampled 
           from {\ty fmin} to {\ty fmax})\\

\hline
\end{tabular*}
\vspace*{.2cm}

When called without output arguments, {\ty tfrbert} runs {\ty tfrqview}
\end{minipage}
\vspace*{1cm}

{\bf \large \sf Example}
\begin{verbatim}
         sig=altes(64,0.1,0.45); 
         tfrbert(sig);
\end{verbatim}
\vspace*{.5cm}

{\bf \large \sf See Also}\\
\hspace*{1.5cm}
\begin{minipage}[t]{13.5cm}
all the {\ty tfr*} functions.
\end{minipage}
\vspace*{.5cm}

{\bf \large \sf References}\\
\hspace*{1.5cm}
\begin{minipage}[t]{13.5cm}
[1] J. Bertrand, P. Bertrand ``Time-Frequency Representations of Broad-Band
Signals'' IEEE ICASSP-88, pp. 2196-2199, New-York, 1988.\\

[2] J. Bertrand, P. Bertrand ``A Class of Affine Wigner Functions with
		  Extended Covariance Properties'', J. Math. Phys.,
		  Vol. 33, No. 7, July 1992.
\end{minipage}
