% This is part of the TFTB Reference Manual.
% Copyright (C) 1996 CNRS (France) and Rice University (US).
% See the file refguide.tex for copying conditions.



\markright{dopnoise}
\section*{\hspace*{-1.6cm} dopnoise}

\vspace*{-.4cm}
\hspace*{-1.6cm}\rule[0in]{16.5cm}{.02cm}
\vspace*{.2cm}



{\bf \large \sf Purpose}\\
\hspace*{1.5cm}
\begin{minipage}[t]{13.5cm}
Complex doppler random signal.
\end{minipage}
\vspace*{.5cm}


{\bf \large \sf Synopsis}\\
\hspace*{1.5cm}
\begin{minipage}[t]{13.5cm}
\begin{verbatim}
[y,iflaw] = dopnoise(N,fs,f0,d,v)
[y,iflaw] = dopnoise(N,fs,f0,d,v,t0)
[y,iflaw] = dopnoise(N,fs,f0,d,v,t0,c)
\end{verbatim}
\end{minipage}
\vspace*{.5cm}


{\bf \large \sf Description}\\
\hspace*{1.5cm}
\begin{minipage}[t]{13.5cm}
        {\ty dopnoise} generates a complex noisy doppler signal, normalized
        so as to be of unit energy. \\

\hspace*{-.5cm}\begin{tabular*}{14cm}{p{1.5cm} p{8.5cm} c}
Name & Description & Default value\\
\hline
         {\ty N } & number of points\\  
         {\ty fs} & sampling frequency (in Hz)\\ 
         {\ty f0} & target frequency   (in Hz)\\  
         {\ty d } & distance from the line to the observer (in meters)\\  
         {\ty v } & target velocity    (in m/s)\\  
         {\ty t0} & time center                &   {\ty N/2}\\ 
         {\ty c}  & wave velocity      (in m/s) &  {\ty 340}\\
   \hline {\ty y}  & output signal\\
         {\ty iflaw} & model used as instantaneous frequency law\\

\hline
\end{tabular*}
\vspace*{.2cm}

{\ty [y,iflaw] = dopnoise(N,fs,f0,d,v,t0,c)} returns the signal received by
a fixed observer from a moving target emitting a random broad-band white
gaussian signal whose central frequency is {\ty f0}. The target is moving
along a straight line, which gets closer to the observer up to a distance
{\ty d}, and then moves away. {\ty t0} is the time center (i.e. the time at
which the target is at the closest distance from the observer), and {\ty c}
is the wave velocity in the medium.

\end{minipage}

\newpage

{\bf \large \sf Example}\\
\hspace*{1.5cm}
\begin{minipage}[t]{13.5cm}
Consider such a noisy doppler signal and estimate its instantaneous
frequency (see {\ty instfreq}) :
\begin{verbatim}
         [z,iflaw]=dopnoise(500,200,60,10,70,128);
         subplot(211);    plot(real(z)); 
         subplot(212);    plot(iflaw);   hold; 
         ifl=instfreq(z); plot(ifl,'g'); hold; 
         sum(abs(z).^2)
         ans = 
               1.0000
\end{verbatim}
The frequency evolution is hardly visible  from the time representation,
whereas the instantaneous frequency estimation shows it with success. We
check that the energy is equal to 1.
\end{minipage}
\vspace*{.5cm}


{\bf \large \sf See Also}\\
\hspace*{1.5cm}
\begin{minipage}[t]{13.5cm}
\begin{verbatim}
doppler, noisecg.
\end{verbatim}
\end{minipage}
