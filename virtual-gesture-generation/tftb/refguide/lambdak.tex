% This is part of the TFTB Reference Manual.
% Copyright (C) 1996 CNRS (France) and Rice University (US).
% See the file refguide.tex for copying conditions.



\markright{lambdak}
\hspace*{-1.6cm}{\Large \bf lambdak}

\vspace*{-.4cm}
\hspace*{-1.6cm}\rule[0in]{16.5cm}{.02cm}
\vspace*{.2cm}



{\bf \large \sf Purpose}\\
\hspace*{1.5cm}
\begin{minipage}[t]{13.5cm}
Evaluate lambda function for Affine Wigner distribution.
\end{minipage}
\vspace*{.5cm}


{\bf \large \sf Synopsis}\\
\hspace*{1.5cm}
\begin{minipage}[t]{13.5cm}
\begin{verbatim}
Y=lambdak(U,K)
\end{minipage}
\vspace*{.5cm}


{\bf \large \sf Description}\\
\hspace*{1.5cm}
\begin{minipage}[t]{13.5cm}
        Y=lambdak(U,K) evaluates the parametrization lambda function
        involved in the affine smoothed pseudo Bertrand distribution.\\

         $LAMBDAK(U,0) = -U/(exp(-U)-1)$ for K = 0
         $LAMBDAK(U,1) = exp(1+U exp(-U)/(exp(-U)-1))$ for K = 1
         $LAMBDAK(U,K) = (K (exp(-U)-1)/(exp(-KU)-1))^(1/(K-1))$ otherwise\\

\hspace*{-.5cm}\begin{tabular*}{14cm}{p{1.5cm} p{8.5cm} c}
Name & Description & Default value\\
\hline
        U & real vector\\
        Y & value of LAMBDAD at point(s) U\\

\hline
\end{tabular*}

\end{minipage}
\vspace*{1cm}


{\bf \large \sf See Also}\\
\hspace*{1.5cm}
\begin{minipage}[t]{13.5cm}
\begin{verbatim}
LAMBDAB, LAMBDAU, LAMBDAD.
\end{minipage}

